\documentclass[dvipdfmx]{beamer}
\usepackage{mytheme}
\setbeamercovered{transparent=0}

\input{header_listing.tex}
\newcommand{\textblue}[1]{\textcolor{blue}{#1}}

\title{Julia デバッガ ``Gallium.jl'' の紹介}
\author{yomichi}
\date{JuliaTokyo \#6 2015/04/25}

\begin{document}

\begin{frame}
    \titlepage
    スライドとサンプルコード \\ 
    https://github.com/yomichi/JuliaTokyo
\end{frame}

\begin{frame}[containsverbatim]
\frametitle{自己紹介}
\begin{columns}
    \begin{column}{.8\linewidth}
        \begin{itemize}
            \item HN : 夜道, yomichi
            \item twitter : \verb|@yomichi_137|
            \item ぽすどくさんねんせい
            \begin{itemize}
                \item 統計物理学・計算物理学
                \begin{itemize}
                    \item 主にモンテカルロ法とか
                    \item 言語はC++, Python, \textblue{Julia}
                \end{itemize}
            \end{itemize}
            \item イベント(勉強会)実況勢
            \begin{itemize}
                \item 最近だと全ゲ連(同人ゲーム開発)とか、PyConJP (Python) とか、\textblue{JuliaTokyo}とか
            \end{itemize}
            \item Julia の開発環境は REPL と Vim
            \item コミケでJulia 本出したりしてます
            \begin{itemize}
                \item 去年の冬に v0.4 入門本を、今年の夏は v0.5 新機能本を出していました
                \item http://yomichi.hateblo.jp/entry/2016/08/11/120654 からDL 可能
            \end{itemize}
        \end{itemize}
    \end{column}
    \begin{column}{.3\linewidth}
        \includegraphics[width=\linewidth]{sakuremi.jpg}

        (thanks \verb|@am_11|)
    \end{column}
\end{columns}
\end{frame}

\begin{frame}[containsverbatim]
    \frametitle{デバッグツール}
    \begin{itemize}
        \item \verb|@show|
        \begin{itemize}
            \item Julia に標準で含まれるマクロ(神)
            \item 変数や式を渡すと、渡したものと、その評価結果との組を標準出力に書き出す
            \begin{lstlisting}
julia> @show sin(pi)
sin(pi) = 1.2246467991473532e-16
1.2246467991473532e-16
            \end{lstlisting}
            \item 自分で書くコードのデバッグはほとんどこれで十分
            \item 他人が書いたもの、特にJulia 本体の動作を探るには、ちょっと面倒くさい
            \begin{itemize}
                \item とくに Julia 本体の場合は再コンパイルが必要
            \end{itemize}
        \end{itemize}
\end{itemize}
\end{frame}

\begin{frame}[containsverbatim]
    \frametitle{デバッグツール}
    \begin{itemize}
        \item \verb|Gallium.jl|
        \begin{itemize}
            \item Julia のJulia によるJulia のためのデバッガ
            \item Julia v0.5 から使える
            \item つくったひと: Keno Fischer (Cxx.jl とかの作者)
            \item マクロを使ってJulia の構文木を直接触ることでデバッガとしての機能を実現
            \begin{itemize}
                \item ステップ実行、変数の値を表示、ブレークポイントの設定、...
            \end{itemize}
            \item ソースコードをいじらずにデバッグ実行できるので手軽
            \item 標準ライブラリや外部パッケージのコードレビューが捗る
            \item Juno というJulia IDE (Atom base) とも連携できるらしいですよ
        \end{itemize}
\end{itemize}
\end{frame}

\begin{frame}
    \frametitle{}
    とりあえずつかってみればいいんじゃないかな(デモするよ、の意)

    公開する資料では簡単なコマンド一覧とかを追加する(はず)
\end{frame}

\end{document}
